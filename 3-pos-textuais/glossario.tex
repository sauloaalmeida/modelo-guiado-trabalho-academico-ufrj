\newpage
\phantomsection
\addcontentsline{toc}{chapter}{GLOSSÁRIO} % Adiciona o glossário
\begin{center}
\textbf{\MakeUppercase{Glossário}}
\end{center}

\noindent\textbf{ABNT:} Associação Brasileira de Normas Técnicas, associação civil de utilidade pública responsável pela elaboração das Normas Brasileiras (ABNT NBR).

\vspace{10pt}\noindent\textbf{Acessibilidade:} possibilidade e condição de alcance, percepção e entendimento para utilização, com segurança e autonomia, de espaços, mobiliários, equipamentos urbanos, edificações, transportes, informação e comunicação, inclusive seus sistemas e tecnologias, bem como outros serviços e instalações abertos ao público, de uso público ou privado de uso coletivo, tanto na zona urbana como na rural, por pessoa com deficiência ou mobilidade reduzida.

\vspace{10pt}\noindent\textbf{Beiral:} prolongamento do telhado que excede a prumada de uma parede externa da edificação.

\vspace{10pt}\noindent\textbf{Cisterna:} reservatório subterrâneo ou no nível do solo utilizado para armazenamento de água.

\vspace{10pt}\noindent\textbf{Cota:} medida de distância entre dois pontos, expressa em centímetros.

\vspace{10pt}\noindent\textbf{Cota de nível} medida de distância vertical em relação à Referência de Nível, expressa em metros.

\vspace{10pt}\noindent\textbf{Declividade:} é a razão ou divisão entre a diferença da altura entre dois pontos considerados e a distância horizontal entre esses pontos, expressa em porcentagem.

\vspace{10pt}\noindent\textbf{Duto de ventilação:} espaço vertical ou horizontal delimitado no interior de uma edificação destinado, entre outros fins à ventilação.

\vspace{10pt}\noindent\textbf{Edícula:} edificação secundária e acessória da moradia, geralmente situada no fundo do lote, que não constitui domicílio independente.

\vspace{10pt}\noindent\textbf{Eixo transversal do lote:} eixo que conecta o meio das divisas em linha reta localizado no ponto médio das mesmas.

\vspace{10pt}\noindent\textbf{Fachada:} elevação das partes externas de uma edificação.

\vspace{10pt}\noindent\textbf{Guarda-corpo:} elemento construído de proteção vertical que delimita as faces laterais de escadas, rampas, patamares, terraços, sacadas, mezaninos, passarelas, e galerias.

\vspace{10pt}\noindent\textbf{Guia ou meio-fio:} borda física instalada ao longo das vias, de acabamento da calçada ou passeio, junto à sarjeta (escoamento pluvial), podendo ser rebaixada em casos de acesso de 
veículos e pedestres.

\vspace{10pt}\noindent\textbf{Logradouro público:} área de terra de propriedade pública e de uso comum e/ou especial  do povo destinada às vias de circulação, às praças e aos espaços livres.

\vspace{10pt}\noindent\textbf{Lote:} terreno oriundo de processo regular de parcelamento do solo, com acesso e  testada para logradouro público, servido de infraestrutura básica.

\vspace{10pt}\noindent\textbf{Mansarda:} abertura ou janela que se projeta além da água do telhado de uma  edificação.

\vspace{10pt}\noindent\textbf{Nível do pavimento térreo (NT):} nível atribuído ao piso acabado do pavimento térreo e  determinado pela Referência de Nível (RN) acrescida de no máximo 1,20m (um metro e vinte centímetros).

\vspace{10pt}\noindent\textbf{Passeio:} parte da via de circulação ou logradouro público destinada ao tráfego de pedestres.

\vspace{10pt}\noindent\textbf{Pavimento:} volume compreendido entre lajes de uma edificação.

\vspace{10pt}\noindent\textbf{Pé direito:} distância vertical entre o piso e o teto de um compartimento.

\vspace{10pt}\noindent\textbf{Rampa:} plano inclinado com declividade igual ou superior a 5\% (cinco por cento) de inclinação que interliga dois níveis distintos de piso.

\vspace{10pt}\noindent\textbf{Recuo frontal ou recuo da testada do lote:} é a menor distancia entre uma edificação e a testada do lote onde se situa, medida perpendicularmente em relação à testada do lote, a partir do ponto mais avançado da edificação.

\vspace{10pt}\noindent\textbf{Reservatório de Retardo:} dispositivo aberto ou fechado capaz de reter parte das águas pluviais e liberá-las de forma controlada nas galerias responsáveis de drenagem pública.

\vspace{10pt}\noindent\textbf{Sótão:} espaço utilizável sob a cobertura inclinada, no qual não se admite a elevação de paredes no perímetro da edificação além daquela necessária à estrutura da própria cobertura.

\vspace{10pt}\noindent\textbf{Subsolo:} pavimento situado em nível abaixo da laje do Nível do pavimento térreo (NT).

\vspace{10pt}\noindent\textbf{Talvegue:} linha sinuosa em fundo de vale por onde correm as águas.

\vspace{10pt}\noindent\textbf{Tapume:} vedação provisória do terreno usada durante a construção.

\vspace{10pt}\noindent\textbf{Vistoria:} diligência determinada na forma deste Código para verificar as condições de uma obra, instalação ou exploração de qualquer natureza.