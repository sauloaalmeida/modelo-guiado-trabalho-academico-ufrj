\chapter{Resultados}\label{cap:resultados}

Essa seção presenta os resultados do experimento \lipsum[11]

\section{Resultados do experimento}

\lipsum[12-13]

Na Tabela \ref{tab:resultados} foram tabulados todos os dados de resultados das execuções dos experimentos. 

A tabela apresenta uma listagem \lipsum[14]

\lipsum[15-17]

\subsection{Análise dos resultados}
Ao avaliar os dados da Tabela \ref{tab:resultados}, é possível realizar algumas observações de cunho mais geral:

\lipsum[18]


\begin{table}[H]
\caption{Tempos treinamento, predição e acurácia}
\begin{center}
\fontsize{10pt}{13pt}\selectfont
\begin{tabular}{lccc}
\hline
Dataset                       & Tipo processamento  &   Tempo Predict (s)    &Acurácia1 (\%) \\
\hline
\textit{Iris}                 & Serial              &  0.0287 (0.005281)     & 94.24 (1.41)  \\
                              & CPU Multiprocesso   &  0.0272 (0.000206)     & 93.22 (0.91)  \\
                              & GPU                 &  0.0279 (0.000537)     & 93.31 (1.64)  \\
\hline
\textit{Sonar}                & Serial              &  0.0718 (0.000702)     & 85.26 (1.67)  \\
                              & CPU Multiprocesso   &  0.0705 (0.000369)     & 86.36 (1.69)  \\
                              & GPU                 &  0.0703 (0.000834)     & 85.80 (1.18)  \\
\hline
\textit{Glass Identification} & Serial              &  0.0575 (0.000698)     & 67.91 (1.97)  \\
                              & CPU Multiprocesso   &  0.0577 (0.000530)     & 68.40 (1.84)  \\
                              & GPU                 &  0.0571 (0.000587)     & 67.66 (1.76)  \\
\hline
\textit{Libras Movement}      & Serial              &  0.2050 (0.001077)     & 80.06 (1.24)  \\
                              & CPU Multiprocesso   &  0.2020 (0.001227)     & 80.59 (1.16)  \\
                              & GPU                 &  0.2022 (0.001252)     & 80.34 (1.03)  \\
\hline
\multicolumn{4}{l}{Fonte: O autor (2025)}
\end{tabular}
\label{tab:resultados}
\end{center}
\end{table}


A medida de aceleração %informação do speedup pode ser 
foi calculada pela Equação~\eqref{eq:speedup}, onde \(A\) é a \textit{aceleração}, \(T_s\) é o tempo de execução serial/sequencial e \(T_p\) é o tempo de execução paralelo. Dessa forma é possível analisar quantas vezes o desempenho em relação ao tempo sequencial foi impactado pela quantidade de processos utilizados. Quanto maior o número, mais rápido o algoritmo foi executado. 

\begin{equation} 
A = T_s / T_p
\label{eq:speedup}
\end{equation}