\chapter{Experimentos}\label{cap:experimentos}

Nesse capítulo será abordado o escopo dos experimentos, em qual ambiente foram executados, os detalhes dos datasets utilizados bem como a metodologia definida para a sua execução. 

\section{Ambiente utilizado}


\subsection{Configurações de hardware}

Os experimentos foram realizados utilizando a seguinte configuração de hardware\index{hardware}:

\begin{itemize}
    \item Processador\index{processador}: Intel\textsuperscript{\tiny{\copyright}} Core\textsuperscript{\tiny{TM}} i9-13900KF 
    \begin{itemize}
        \item Núcleos: 20
        \item Threads: 32
        \item Velocidade: 5.40GHz
    \end{itemize}
    \item Memória RAM: 64 GB
    \item SSD M2: 1.8 TB
    \item Placa de vídeo: NVIDIA GeForce RTX 4060, 8GB
\end{itemize}

\subsection{Configurações de software}

Os seguintes configurações e componentes de software\index{software} foram utilizados durante a sua execução:

\begin{itemize}
    \item Sistema Operacional: Linux Mint 22, kernel 6.8.0-41-generic
    \item Python (3.12.8) 
    \item Conda (24.11.3) 
    \item Matplotlib (3.9)
\end{itemize}

\section{Exemplo de Section do experimento}

\lipsum[7]

\begin{equation} 
DP = (T_p - T_s) / T_s * 100
\label{eq:diferenca_percentual}
\end{equation}

A equação \eqref{eq:diferenca_percentual} apresentada em a diferença percentual entre o tempo serial e o tempo paralelo de execução. 

\lipsum[1-2]



\subsection{Exemplo de subsection do experimento}

\lipsum[8]
A Tabela \ref{tab:tabelaexemplo1}, apresenta as informações \lipsum[9]

Foram descartadas as observações que possuíam valores nulos, e os atributos que não não eram do tipo numérico.  
\begin{table}[H]
\caption{Exemplo de tabela de uso genérico}
\begin{center}
\fontsize{10pt}{13pt}\selectfont
\begin{tabular}{lcccc}
\toprule
                  Base de dados & Observações  &   Features   &   Classe      & Desbalanceamento\\
                                & (used/total) & (used/total) & (used/total)  &                 \\
\midrule
                       Dataset1 &   150/150    &     4/4      &     3/3       & 1.00            \\
                       Dataset3 &   208/208    &    60/60     &     2/2       & 1.14            \\
                       Dataset3 &   214/214    &     9/10     &     6/7       & 8.44            \\
\bottomrule
\multicolumn{5}{l}{Fonte: O autor (2025)}
\end{tabular}
\label{tab:tabelaexemplo1}
\end{center}
\end{table}


\subsubsection{Exemplo de subsubsection do experimento}

\lipsum[10-12]
