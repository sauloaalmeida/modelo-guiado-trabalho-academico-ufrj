\chapter{Introdução}

\section{Motivação ou problemática}

Relatar o que motivou a executar essa pesquisa, seja uma motivação da sua experiência pessoal ou uma motivação com base em problemas sociais atuais.

\section{Objetivos da pesquisa}

Nesta parte, você deve salientar qual o fenômeno a ser investigado e quais os objetivos principais e secundários ao pesquisar esse fenômeno.

Esta monografia visa responder qual pergunta? É importante que o aluno deixe claro a pergunta da pesquisa.

O objetivo pode ser organizado em Objetivo Geral e Objetivos Específicos, caso você queira falar do objetivo final e das partes que você precisará fazer para atingi-lo. Ou pode ser organizado em Objetivo Primário e Objetivos secundários, onde você fala do seu objetivo principal e de outros que você alcançará após a realização do primeiro.

\section{Relevância}

Por que é importante estudar esse tema? Qual a relevância acadêmica e prática de investigar tal assunto? 

A relevância fica diminuída quando apenas o próprio autor da monografia parece considerá-la importante. Assim, preferencialmente, a relevância deve ser levantada a partir da leitura de outros autores. Cite outros autores que consideram importante o estudo de tal tema.

\section{Delimitação da pesquisa}

A seção de delimitação de estudo visa demarcar o escopo de sua pesquisa, indicando ao leitor partes que não serão o alvo da sua investigação.

\section{Estrutura do trabalho}

O restante deste texto está organizado da seguinte forma. A Seção~\ref{cap:referencial_teorico} apresenta os conceitos básicos do referencial teórico necessário para um melhor entendimento do trabalho, com a apresentação dos assuntos tal, tal e tal. A Seção~\ref{cap:metodologia} relata qual foi a metodologia utilizada no trabalho. A Seção~\ref{cap:experimentos} explica como os experimentos foram executados. A Seção~\ref{cap:resultados} apresenta os resultados dos experimentos, Por fim a Seção~\ref{cap:conclusao} apresenta um sumário dos resultados e sugestões para trabalhos futuros.