\addtocontents{toc}{\protect\thispagestyle{empty}} % força o "empty" na primeira pagina do sumário
\pagestyle{empty} % remove a nunemracao das demais paginas de sumario (se houver mais de uma) 

%apresenta ate o nivel 5, de paragrafo
\setcounter{tocdepth}{4}

%apresenta a numeracao ate o nivel 5, de paragrafo
\setcounter{secnumdepth}{4}

% titletoc - Configuração da formatação dos títulos (em seus vários níveis) no sumário. 
% A formatação desses mesmos títulos dentro do texto, precisa ser feito antes do begin{document} e ficou no arquivo: config/configuracoes.tex
\titlecontents{chapter}[12pt]
  {\normalfont\normalsize\bfseries}
  {\contentslabel{12pt}\uppercase} % removido \MakeUppercase
  {}
  {\titlerule*[0.4pc]{.}\contentspage} % aqui sim!

\titlecontents{section}[20pt]
  {\normalfont\normalsize}
  {\contentslabel{20pt}\uppercase}
  {}
  {\titlerule*[0.4pc]{.}\contentspage}

\titlecontents{subsection}[32pt]
  {\normalfont\normalsize\bfseries}
  {\contentslabel{32pt}}
  {}
  {\titlerule*[0.4pc]{.}\contentspage}

\titlecontents{subsubsection}[38pt]
  {\normalfont\normalsize}
  {\contentslabel{38pt}}
  {}
  {\titlerule*[0.4pc]{.}\contentspage}

\titlecontents{paragraph}[46pt]
  {\normalfont\normalsize}
  {\contentslabel{46pt}}
  {}
  {\titlerule*[0.4pc]{.}\contentspage}


%Altera e Formata o titulo do sumario
\renewcommand{\contentsname}{\normalsize\centerline{\bfseries\uppercase{Sumário}}}

\tableofcontents % Insere o sumário
\clearpage % Garante que o sumário termine aqui
\pagestyle{fancy} % Retorna ao estilo "fancy" para o restante do documents